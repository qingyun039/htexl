% 拦腰截取pdf页面的一部分
\newcommand{\includepdfpart}[4]{% 参数为: 页面,上距,下距,pdf文件名
	\par\noindent
	\makebox[\textwidth]{\includegraphics[page=#1,trim=0 #3 0 #2,clip]{#4}}
	\par
}

% 画雷达图
%\usepackage{tikz}
%\usetikzlibrary{calc}

\pgfkeys{
	/rada/.is family, /rada,
	default/.style = {    % 默认参数
		max=100, min=40, level=7, radius=5.5, color=blue
	},
	max/.estore in = \max,        % 传递的参数保存于此宏中
	min/.estore in = \min,
	level/.estore in = \level,
	radius/.estore in = \radius,
	color/.estore in = \mainclr,
}
\newcommand{\rada}[3][]{
	\pgfkeys{/rada, default, #1}
	\def\alist{#2}   % 可能合并成一个参数更好,较不易出错
	\def\blist{#3}
	\pgfmathsetmacro{\step}{(\max-\min)/(\level-1)}   % 计算结果保存于此宏中
	\pgfmathsetmacro{\nextv}{\min+\step}
	\pgfmathsetmacro{\leveln}{0.25+0.75/(\level-1)}
	\begin{tikzpicture}[every node/.style={font=\sffamily\color{gray}}]
	% 得到项数
	\foreach \x[count=\xi] in \alist {\global\let\maxitems\xi}
	
	% 可能用 let 比较好,
	% let 和 def 的区别(摘自TeX book的习题和答案):
	% EXERCISE 20.8: Is there a significant difference between '\let\a=\b' and '\def\a{\b}' ?
	%
	% ANSWER: Yes indeed. In the first case, \a receives the meaning of \b that is current at the time of the \let. 
	% In the second case, \a becomes a macro that will expand into the token \b whenever \a is used, 
	% so it has the meaning of \b that is current at the time of use. You need \let, if you want to inter-
	% change the meanings of \a and \b. 
	\global\def\ang{360/\maxitems}
	\global\def\startang{270+\ang/2}
	
	% 每项位置
	\foreach[count=\ii] \s[evaluate=\s as \x using (\startang+\ii*\ang)] in \alist{
		\coordinate (c\ii) at (\x:\radius);
		\node[label=\x:\s] at (c\ii) {};
	}

	% 画网
	\coordinate (v) at (0:0);
	\foreach[count=\ii] \f in {0.25,\leveln,...,1}{
		\draw[color=\mainclr] ($(v)!\f!(c1)$) foreach \i in {2,3,...,\maxitems} { -- ($(v)!\f!(c\i)$) } -- cycle;
		\coordinate (h\ii) at ($(v)!\f!(c\maxitems)$);
	}
	
	% 刻度
	\foreach[count=\i] \j in {\min,\nextv,...,\max}{
		\pgfmathreal{\j}
		\node[above] at (h\i -| v) {\pgfmathresult};
	}

	% 阴影部分
	\foreach[count=\ii] \st[evaluate=\st as \s using (0.75*(\st-\min)/(\max-\min)+0.25)] in \blist {
		\coordinate (s\ii) at ($(v)!\s!(c\ii)$);
		\fill[fill=\mainclr!70!gray] (s\ii) circle (3pt);
		\node[circle,sloped] at ($(s\ii)!12pt!(c\ii)$) {\st};
	}
	\foreach[count=\b] \a in {2,3,...,\maxitems,1}{ \coordinate (tmp\b) at ($(s\b)!.5!(s\a)$); }
	\foreach \i in {1,2,...,\maxitems}{
		\coordinate (d\i) at ($(tmp\i)!.5cm!-90:(s\i)$); 
	}
	\fill[fill=\mainclr!70, opacity=.7] (s1) foreach[count=\j] \i in {2,3,...,\maxitems}{ ..controls(d\j).. (s\i) } .. controls(d\maxitems) .. cycle;	
	\end{tikzpicture}
}
